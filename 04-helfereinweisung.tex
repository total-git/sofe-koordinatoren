\section{Helfereinweisung}
Diese Einweisung sollte jede/r, der an eurer Theke hilft, zu Beginn seiner/ihrer Schicht bekommen.
\subsection{Verkauf und Pfand}
\begin{itemize}
  \renewcommand{\labelitemi}{$\Box$}
  \item Es gibt kein Bargeld an der Theke und es kann nicht bar bezahlt werden. Es müssen Marken am Markenverkauf gekauft werden.
  \item Marken nach Annahme zählen und wenn passend zerreißen und wegschmeißen.
  \item Es wird \textbf{nicht} nachgeschenkt. Die Alkoholmenge ist mit dem Dosierer genau festgelegt: \SI{5}{\centi\litre} bei den Longdrinks mit Wodka, Gin und Rum bzw.\ \SI{10}{\centi\litre} bei Malibu-Kirsch.
  \item Non-Alk wird in der Plastikflasche ausgegeben, auch 1 Marke Pfand.
  \item Zu allen Getränkepreisen kommt 1 Marke Pfand dazu und es wird eine \textbf{Pfandmarke} ausgegeben!
    
    Wenn ein Becher zurückgegeben wurde, nur den Becher zurücknehmen, \textbf{keinen} Pfand verlangen und auch \textbf{keinen} Pfandbon ausgeben (der Gast hat ja noch seinen Pfandbon).
  \item Pfand können Gäste an der Theke nicht zurückbekommen, weder bar noch als Marken. Pfandrückgabe gibt es nur an der Markenausgabe.
  \item \textbf{Preisliste zeigen}
\end{itemize}
% ---
\subsection{T-Shirts}
\begin{itemize}
  \renewcommand{\labelitemi}{$\Box$}
  \item Helfer-T-Shirt hinter der Theke anhaben.
  \item Helfer-T-Shirt außerhalb der Schicht ausziehen!
  \item T-Shirt Farben: Helfer (grün), Koordinatoren (blau) und Organisatoren (lila bzw.\ grau für Pullis)
\end{itemize}
% ---
\subsection{Helfergetränke}
\begin{itemize}
  \renewcommand{\labelitemi}{$\Box$}
  \item Während der Schicht sind Helfergetränke kostenlos, davor und danach Marken benutzen.
  \item Helfergetränke für Helfer \textbf{unserer Theke} werden in einer Strichliste eingetragen.
  \item Alle die Getränke kostenlos bekommen, haben Marken. \textbf{Niemand} bekommt Getränke einfach so.
  \item Helfer zahlen keinen Pfand und bekommen auch keine Pfandmarke, zahlen also nur die passende Menge an Helfermarken: \textbf{Zeigen, wie Helfermarken aussehen}
\end{itemize}
% ---
\subsection{Mischen und Zapfen}
\begin{itemize}
  \renewcommand{\labelitemi}{$\Box$}
  \item \textbf{Zeigen, wie man Longdrink mischt und Bier zapft!}
  \item Auf den Boden gefallene Strohhalme etc. in den Müll
\end{itemize}
% ---
\subsection{Sicherheit}
\begin{itemize}
  \renewcommand{\labelitemi}{$\Box$}
  \item Bei Problemen (Schlägerei, pöbelnde Kunden, nicht bezahlte Getränke etc.) Koordinatoren holen, dann rufen sie Orgas/Securities. \textbf{Nicht} selber dazwischen gehen!
  \item \textbf{Fluchtwege zeigen!}
  \item Feuerlöscher stehen bereit: \textbf{Zeigen, wo hinter der Theke}.
    
    Bei Feuer: \textbf{Erst} Orgas verständigen bzw.\ Feuerwehr rufen, \textbf{dann} versuchen, das Feuer zu löschen:
    
    Stift aus dem Griff ziehen und Griff zusammendrücken. \textbf{Kurze} Sprühstöße mit dem Löscher
  \item Bei Verletzungen Erste Hilfe leisten und Sanis rufen (über Orgas oder Securities)
  \item \textbf{Niemand} außer Orgas (dunkelrote Shirts) und unseren Thekenhelfer darf hinter die Theke. Bei Zweifel fragen bzw.\ Koordinatoren rufen.
\end{itemize}
