\section{Nach dem Fest}
% ---
\subsection{Inventur}
Zählt bitte \textbf{alle} übrigen Getränke und tragt sie in die angehängte Liste (\emph{Nachher}) ein. Gebt die Mengen in \textbf{Kästen} (Non-Alk, Bier), \textbf{Fässern} (Bier, Cider) und \textbf{Kartons} (Schnaps, Prosecco etc.) an.

Gebt diese Liste zusammen mit der Liste mit den Getränken der Thekenhelfer einem der Getränkeorgas: Felix, Markus oder Christoph.
% ---
\subsection{Getränke}
\begin{itemize}
  \item Übrige, noch geschlossene, Schachteln mit Schnapsflaschen bitte nach Absprache mit einem Getränke-Orga verräumen.
  \item Offene Flaschen an andere, noch offene Theken bringen, falls möglich.
  \item Eis über dem nächsten Gulli ausleeren. Die \textbf{Styroporkisten} bitte nicht kaputtmachen, dafür zahlen wir Pfand!
\end{itemize}
% ---
\subsection{Paletten}
\begin{itemize}
  \item Biertischgarnituren: \begin{itemize}
      \item Je 20 Biertischgarnituren werden auf eine \textbf{große} Palette gestapelt.
      \item Pro Ebene liegt eine Garnitur (1 Tisch und 2 Bänke).
      \item \textbf{Alle} Tische und Bänke mit der Oberseite nach \textbf{unten}.
      \item Jeweils \textbf{Tische auf Bänke} und \textbf{Bänke auf Tische} stapeln, also immer abwechselnd.
      \item Danach sichern wir alles mit Palettenband. Ihr müsst also keine Spanngurte o.ä.\ verwenden.
    \end{itemize}
  \item Fässer und Kästen -- soweit möglich -- nach voll und leer getrennt auf Paletten.
  \item Becherkisten sortieren nach
    \begin{itemize}
      \item \SI{0.3}{\litre} oder \SI{0.5}{\litre}
      \item benutzt oder unbenutzt
    \end{itemize}
    Insgesamt also 4 Stapel auf getrennten Paletten!
  \item Alle kleinen Paletten bitte mit der \textbf{schmalen} Seite zum Weg, sodass sie direkt mit dem Stapler erreichbar sind.
  \item Alle großen Paletten mit Biertischgarnituren bitte mit der \textbf{breiten} Seite zum Weg.
\end{itemize}
% ---
\subsection{Equipment}
\begin{itemize}
  \item Jeweils 2 \textbf{Kühlschränke} auf eine kleine Palette und mit dem mitgelieferten Band sichern.
  \item \textbf{Zapfanlagen} in Kisten verpacken.
    
    Falls ohne Kiste geliefert, jeweils 4 Zapfen auf eine kleine Palette stellen.
  \item \textbf{CO$_2$-Flaschen} zurück in den Müllgang tragen. Vorsicht, nicht fallen lassen!
  \item \textbf{Müll} im Müllgang in die entsprechenden Mülltonnen entsorgen: Restmüll, Glasflaschen \textbf{ohne} Pfand (Schnaps u.ä.), Papier
  \item \textbf{Feuerlöscher} ins Möbellager bringen.
  \item \textbf{Pfandmarken} zurück in die Thekenkisten.
    
    \textbf{Thekenkisten und Helferkisten} wieder zusammenpacken und ins Lager (Raum A120) bringen.
  \item \textbf{Pavillons} stehen lassen.
\end{itemize}
% ---
\subsection{Heimgehen und ausschlafen \smiley}
Vielen Dank für eure Hilfe!
