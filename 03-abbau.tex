\section{Nach dem Fest}
% ---
\subsection{Inventur}
\textbf{Alle} übrigen Getränke zählen und in die angehängte Liste (\emph{Nachher}) eintragen. Gebt hier die Mengen in \textbf{Kästen} (Non-Alk, Bier), \textbf{Fässern} (Bier, Cider) und \textbf{Flaschen} (Schnaps, Prosecco etc.) an.

Diese Liste zusammen mit der Liste \emph{Vorher} und der Liste mit den Getränken der Thekenhelfer einem der Getränkeorgas (Felix, Markus oder Christoph) geben.
% ---
\subsection{Getränke}
\begin{itemize}
  \item Übrige, noch geschlossene, Schachteln mit Schnapsflaschen bitte nach Absprache mit einem Getränke-Orga verräumen.
  \item Offene Flaschen an andere, noch offene Theken bringen, falls möglich.
  \item Eis über dem nächsten Gulli ausleeren. Die \textbf{Styroporkisten} haben Pfand!
\end{itemize}
% ---
\subsection{Paletten}
\begin{itemize}
  \item Biertischgarnituren:
    \begin{itemize}
      \item Je 20 Biertischgarnituren werden auf eine \textbf{große} Palette gestapelt.
      \item Pro Ebene liegt eine Garnitur (1 Tisch und 2 Bänke).
      %\item Die \textbf{unterste} Ebene mit der Oberseite nach \textbf{oben}.
      %\item \textbf{Alle anderen} mit der Oberseite nach \textbf{unten}.
      \item \textbf{Alle} Tische und Bänke mit der Oberseite nach \textbf{unten}.
      %\item Jeweils Tische auf Tische und Bänke auf Bänke stapeln.
      \item Jeweils Tische auf Bänke und Bänke auf Tische stapeln.
      \item Danach alles mit Spanngurten sichern. % TODO Palettenband??
    \end{itemize}
  \item Fässer (soweit möglich) nach voll und leer getrennt auf Paletten.
  \item Kästen (soweit möglich) nach voll und leer getrennt auf Paletten.
  \item Becherkisten sortieren nach
    \begin{itemize}
      \item 0,3l oder 0,5l
      \item benutzt oder unbenutzt
    \end{itemize}
    Insgesamt also 4 Stapel auf getrennten Paletten!
  \item Alle kleinen Paletten bitte mit der \textbf{schmalen} Seite zum Weg, sodass sie direkt mit dem Stapler erreichbar sind.
  \item Alle großen Paletten mit Biertischgarnituren bitte mit der \textbf{breiten} Seite zum Weg.
\end{itemize}
% ---
\subsection{Equipment}
% TODO
