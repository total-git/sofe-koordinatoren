\section{Während des Fests}
\subsection{Markensystem}
\begin{itemize}
  \item Wert einer Marke: {\large\textbf{1,25\EUR}}
  \item Getränkepreise:

    \begin{tabular}{lll}
      Non-Alk & 1 Marke & + 1 Marke Pfand \\
      Bier, Cider, Energydrink & 2 Marken & + 1 Marke Pfand \\
      Longdrinks & 3 Marken & + 1 Marke Pfand \\
      Cocktails & 4 Marken & + 1 Marke Pfand
    \end{tabular}
  \item Wenn 1 Marke Pfand bezahlt wurde, \textbf{muss} zusammen mit dem Becher (bzw. Flasche, Dose) \textbf{1 Pfandmarke} (pink) ausgegeben werden.
  \item \textbf{Alle} Getränke werden mit Marken bezahlt. Wer Freigetränke hat (Helfer, Orgas, Musiker etc.), hat genug Marken bekommen.
  \item An der Theke gibt es \textbf{kein} Bargeld!
    \begin{itemize}
      \item Getränke können nur mit Marken bezahlt werden, \textbf{nicht} in bar.
      \item Marken können an der Theke \textbf{nicht} wieder in Bargeld umgetauscht werden.
      \item Geld zurück gibt es \textbf{nur} am Markenverkauf!
    \end{itemize}
  \item Alle Marken werden \textbf{zerrissen und entsorgt}, sobald sie als Bezahlung angenommen und gezählt wurden!
\end{itemize}
% ---
\subsection{Helfergetränke}
\begin{itemize}
  \item Es gibt eigene Helfermarken, die anders aussehen als normale Wertmarken.
  \item Alle, die mit Helfermarken zahlen, geben \textbf{keine} Wertmarke für Pfand ab und bekommen auch \textbf{keine} Pfandmarke.
  \item Für hinter der Theke verbrauchte Getränke müssen keine Helfermarken gezahlt werden. Bitte tragt diese in die dafür vorgesehene Strichliste ein.
    % TODO Helfer T-Shirts kontrollieren! 
\end{itemize}
% ---
\subsection{Ein Gast kommt an die Theke}
\begin{enumerate}
  \item Der Gast hat noch keinen Becher:
    \begin{itemize}
      \item Er/Sie bezahlt den \textbf{Getränkepreis + 1 Marke Pfand}.
      \item Er/Sie bekommt dafür das Getränk \textbf{+ 1 Pfandmarke (pink)}.
    \end{itemize}
  \item Der Gast hat schon einen Becher:
    \begin{itemize}
      \item Er/Sie gibt den Becher ab und bezahlt das Getränk -- \textbf{keine} Marke extra für Pfand.
      \item Er/Sie bekommt ein neues Getränk und \textbf{keine} Pfandmarke -- er/sie hat ja von vorher noch eine.
    \end{itemize}
  \item Der Gast ist ein Helfer/Koordinator/Orga:
    \begin{itemize}
      \item Er/Sie gibt die passende Menge Helfermarken ab -- \textbf{keine} Marke extra für Pfand (egal ob ein Becher zurückgegeben wurde oder nicht).
      \item Er/Sie bekommt dafür das Getränk und \textbf{keine} Pfandmarke.
    \end{itemize}
\end{enumerate}
% ---
\subsection{Getränke}
\subsubsection{Bier und Cider}
\begin{itemize}
  \item Bier (Augustiner Hell) in 50l-Fässern
  \item Cider (Strongbow) in 30l-Fässern % TODO Strongbow?
  \item Bier, Radler und Cider werden in 0,5l-Bechern ausgeschenkt.
  \item Nach Zapfschluss wird kein neues Fass Bier mehr angebrochen. Stattdessen wird Bier aus den Kästen \textbf{in Becher umgefüllt}. Glasflaschen werden \textbf{nicht} ausgegeben.
\end{itemize}
% ---
\subsubsection{Non-Alk}
\begin{itemize}
  \item Es gibt Cola, Spezi, Apfelschorle, Zitronenlimo und Wasser (wenig und viel Kohlensäure).
  \item Non-Alk wird \textbf{ohne} Becher in der PET-Flasche verkauft. Wie bei allen Getränken kommt eine Marke für den \textbf{Pfand} dazu und eine Pfandmarke wird ausgegeben.
\end{itemize}
% ---
\subsubsection{Longdrinks}
% TODO
% ---
\subsubsection{Cocktails}
% TODO
