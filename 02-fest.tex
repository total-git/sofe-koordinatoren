\section{Während des Fests}
\subsection{Bändchen}
Dieses Jahr zum erstem Mal gibt es auch für Gäste Bändchen. Diese sind \textbf{weiß}. Außerdem gibt es Helferbändchen (auch für Koordinatoren) in \textbf{orange}, Orga-Bändchen in \textbf{schwarz} und Künstlerbändchen in \textbf{pink}.

Wer kein Bändchen hat, darf nicht auf dem Fest sein!
\subsection{Markensystem}
\begin{itemize}
    \item Wert einer Marke: {\large\SI{1.25}{\EUR}}
    \item Wertmarken sind \textbf{orange}, Helfermarken sind \textbf{blau} und eher quadratisch.
    \item Getränkepreise:

        \begin{tabular}{lll}
            Non-Alk & 1 Marke & + 1 Marke Pfand \\
            Bier, Cider, Energydrink & 2 Marken & + 1 Marke Pfand \\
            Longdrinks & 3 Marken & + 1 Marke Pfand \\
            Cocktails & 4 Marken & + 1 Marke Pfand
        \end{tabular}
    \item Wenn 1 Marke Pfand bezahlt wurde, \textbf{muss} zusammen mit dem Becher (bzw. Flasche) \textbf{1 Pfandmarke} (pink) ausgegeben werden.
    \item \textbf{Alle} Getränke werden mit Marken bezahlt. Wer Freigetränke bekommt (Helfer, Orgas, Musiker etc.), hat von uns genug Helfermarken bekommen.
    \item \textbf{Es gehen keine Getränke ohne Bezahlung durch Wert- oder Helfermarken über die Theke!}
    \item An der Theke gibt es \textbf{kein} Bargeld!
        \begin{itemize}
            \item Getränke können nur mit Marken bezahlt werden, \textbf{nicht} in bar.
            \item Marken können an der Theke \textbf{nicht} wieder in Bargeld umgetauscht werden.
            \item Geld zurück, z.B.\ Pfand, gibt es \textbf{nur} am Markenverkauf!
        \end{itemize}
    \item Alle Marken werden \textbf{zerrissen und entsorgt}, sobald sie als Bezahlung angenommen und gezählt wurden!
\end{itemize}
% ---
\subsection{Helfergetränke}
\begin{itemize}
    \item Es gibt eigene Helfermarken (blau, quadratisch).
    \item Alle, die mit Helfermarken zahlen, geben \textbf{keine Wertmarke} für Pfand ab und bekommen auch \textbf{keine Pfandmarke}.
    \item Für hinter der Theke verbrauchte Getränke müssen keine Helfermarken gezahlt werden. Bitte tragt diese in die dafür vorgesehene Strichliste ein.
    \item Gäste (weißes Bändchen) bekommen mit Helfermarken \textbf{keine} Getränke.
\end{itemize}
% ---
\subsection{Ein Gast kommt an die Theke}
\begin{enumerate}
    \item Der Gast hat noch keinen Becher:
        \begin{itemize}
            \item Er/Sie bezahlt den \textbf{Getränkepreis + 1 Marke Pfand}.
            \item Er/Sie bekommt dafür das Getränk \textbf{+ 1 Pfandmarke (pinke Visitenkarte)}.
        \end{itemize}
    \item Der Gast hat schon einen Becher:
        \begin{itemize}
            \item Er/Sie gibt den Becher ab und bezahlt das Getränk -- \textbf{keine} Marke extra für Pfand.
            \item Er/Sie bekommt ein neues Getränk und \textbf{keine} Pfandmarke -- er/sie hat ja von vorher noch eine.
        \end{itemize}
    \item Der Gast ist ein Helfer/Koordinator/Orga:
        \begin{itemize}
            \item Er/Sie gibt die passende Menge Helfermarken ab -- \textbf{keine} Marke extra für Pfand (egal ob ein Becher zurückgegeben wurde oder nicht).
            \item Er/Sie bekommt dafür das Getränk und \textbf{keine} Pfandmarke.
        \end{itemize}
\end{enumerate}
% ---
\subsection{Getränke}
\subsubsection{Bier und Cider}
\begin{itemize}
    \item Bier (Augustiner Hell) in \SI{50}{\litre}-Fässern
    \item Cider (Strongbow) in \SI{30}{\litre}-Fässern
    \item Bier, Radler und Cider werden in \SI{0.5}{\litre}-Bechern ausgeschenkt.
    \item Nach Zapfschluss wird kein neues Fass Bier mehr angebrochen. Stattdessen wird Bier aus den Kästen \textbf{in Becher umgefüllt}. Glasflaschen werden \textbf{nicht} ausgegeben.
\end{itemize}
% ---
\subsubsection{Energy Drink}
\begin{itemize}
    \item Es gibt, anders als in den letzten Jahren, \textbf{keinen} Energy Drink mehr in Dosen, sondern in Flaschen.
    \item Wird mit etwas Eis im Becher (\SI{0.3}{\litre}) verkauft.
    \item Kostet \textbf{2} Marken, obwohl es Non-Alk ist.
\end{itemize}
% ---
\subsubsection{Non-Alk}
\begin{itemize}
    \item Es gibt Cola, Spezi, Apfelschorle, Zitronenlimo, Orangenlimo und Wasser (in zwei Sorten: still und medium).
    \item Non-Alk wird \textbf{ohne} Becher in der PET-Flasche verkauft. Wie bei allen Getränken kommt eine Marke für den \textbf{Pfand} dazu und eine Pfandmarke wird ausgegeben.
    \item Es gibt Cola und Wasser in zwei Versionen:
        \begin{itemize}
            \item \SI{0.5}{\litre}-Flasche von Aloisius-Quelle $\rightarrow$ wird als Non-Alk verkauft
            \item \SI{1}{\litre}-Flasche von Coca Cola bzw. Adelholzener $\rightarrow$ wird \textbf{nur} zum Mischen von Rum-Cola bzw.\ Prosecco Aperol verwendet
        \end{itemize}
\end{itemize}
% ---
\subsubsection{Longdrinks}
\begin{enumerate}
    \item ca.\ \SI{100}{\gram} \textbf{Eis} mit der dafür vorgesehenen Schaufel in einen frischen \SI{0.3}{\litre}-Becher
    \item den \textbf{Alkohol} mit dem Dosierer in der richtigen Menge einschenken.

        Der Dosierer muss dafür im richtigen Winkel gehalten werden: Haltet den Hals des Dosierers senkrecht, die Flasche ist dann schräg! Dann stoppt der Dosierer nach \SI{5}{\centi\litre}. Um erneut \SI{5}{\centi\litre} einzuschenken, muss die Flasche einmal aufgerichtet werden.
    \item mit \textbf{Non-Alk} auffüllen
    \item einen \textbf{Strohhalm} dazu
\end{enumerate}

\paragraph{Mischverhältnisse}
\begin{center}
    \begin{tabular}{llll}
        Longdrink & Alkohol & Non-Alk & Eis \\ \hline\hline
        Wodka-Orange &  \SI{5}{\centi\litre} Wodka & \SI{15}{\centi\litre} Orangensaft & $\sim$\SI{100}{\gram} \\ \hline
        Wodka-Lemon & \SI{5}{\centi\litre} Wodka & \SI{15}{\centi\litre} Bitter Lemon & $\sim$\SI{100}{\gram} \\ \hline
        Wodka-Energy & \SI{5}{\centi\litre} Wodka & \SI{15}{\centi\litre} Free Energy & $\sim$\SI{100}{\gram} \\ \hline
        Free Energy & -- & \SI{30}{\centi\litre} (kleiner Becher) & $\sim$\SI{100}{\gram} \\ \hline
        Rum-Cola & \SI{5}{\centi\litre} Rum & \SI{15}{\centi\litre} Coca Cola & $\sim$\SI{100}{\gram} \\ \hline
        Malibu-Kirsch & \SI{10}{\centi\litre} Malibu & 10cl Kirschsaft & $\sim$\SI{100}{\gram} \\ \hline
        Gin-Tonic & \SI{5}{\centi\litre} Gin & \SI{15}{\centi\litre} Tonic Water & $\sim$\SI{100}{\gram} \\ \hline
        Prosecco Aperol & \SI{5}{\centi\litre} Aperol \& \SI{10}{\centi\litre} Proseco & \SI{3}{\centi\litre} Mineralwasser & $\sim$\SI{100}{\gram} \\ \hline
        Hugo & Fertig ca. \SI{20}{\centi\litre} & $\sim$3 Minzblätter
    \end{tabular}
\end{center}

\paragraph{Alkoholsorten}
\begin{itemize}
    \item Wodka: Smirnoff Red Label (\SI{1}{\litre}-Flasche)
    \item Rum: Havanna Club 3 Jahre (\SI{1}{\litre}-Flasche)
    \item Malibu (\SI{1}{\litre}-Flasche)
    \item Gin: Gordon's Dry Gin (\SI{1}{\litre}-Flasche)
    \item Prosecco (\SI{0.7}{\litre}-Flasche)
    \item Aperol (\SI{1}{\litre}-Flasche)
\end{itemize}

% ---
\subsubsection{Cocktails}
Es gibt folgende Cocktails. Die Mischanleitungen werden an der Theke verteilt.
\begin{itemize}
    \item Caipirinha
    \item Long Island Ice Tea
    \item Solero
    \item Tequila Sunrise
    \item Touchdown
    \item Papa Joes
    \item Sex on the Beach
    \item Non-Alk-Cocktail
\end{itemize}
