\section{Während des Fests}
\subsection{Markensystem}
\begin{itemize}
  \item Wert einer Marke: {\large\SI{1.25}{\EUR}}
  \item Wertmarken sind \textbf{grün}, Helfermarken sind \textbf{orange} und \textbf{quadratischer} als die Wertmarken.
  \item Getränkepreise:

    \begin{tabular}{lll}
      Non-Alk & 1 Marke & + 1 Marke Pfand \\
      Bier, Cider, Energydrink & 2 Marken & + 1 Marke Pfand \\
      Longdrinks & 3 Marken & + 1 Marke Pfand \\
      Cocktails & 4 Marken & + 1 Marke Pfand
    \end{tabular}
  \item Wenn 1 Marke Pfand bezahlt wurde, \textbf{muss} zusammen mit dem Becher (bzw. Flasche) \textbf{1 Pfandmarke} (pink) ausgegeben werden.
  \item \textbf{Alle} Getränke werden mit Marken bezahlt. Wer Freigetränke bekommt (Helfer, Orgas, Musiker etc.), hat von uns genug Helfermarken bekommen. Es gehen \textbf{keine} Getränke ohne Bezahlung durch Wert- oder Helfermarken über die Theke!
  \item An der Theke gibt es \textbf{kein} Bargeld!
    \begin{itemize}
      \item Getränke können nur mit Marken bezahlt werden, \textbf{nicht} in bar.
      \item Marken können an der Theke \textbf{nicht} wieder in Bargeld umgetauscht werden.
      \item Geld zurück, z.B.\ Pfand, gibt es \textbf{nur} am Markenverkauf!
    \end{itemize}
  \item Alle Marken werden \textbf{zerrissen und entsorgt}, sobald sie als Bezahlung angenommen und gezählt wurden!
\end{itemize}
% ---
\subsection{Helfergetränke}
\begin{itemize}
  \item Es gibt eigene Helfermarken (orange).
  \item Alle, die mit Helfermarken zahlen, geben \textbf{keine} Wertmarke für Pfand ab und bekommen auch \textbf{keine} Pfandmarke.
  \item Für hinter der Theke verbrauchte Getränke müssen keine Helfermarken gezahlt werden. Bitte tragt diese in die dafür vorgesehene Strichliste ein.
\end{itemize}
% ---
\subsection{Ein Gast kommt an die Theke}
\begin{enumerate}
  \item Der Gast hat noch keinen Becher:
    \begin{itemize}
      \item Er/Sie bezahlt den \textbf{Getränkepreis + 1 Marke Pfand}.
      \item Er/Sie bekommt dafür das Getränk \textbf{+ 1 Pfandmarke (pink)}.
    \end{itemize}
  \item Der Gast hat schon einen Becher:
    \begin{itemize}
      \item Er/Sie gibt den Becher ab und bezahlt das Getränk -- \textbf{keine} Marke extra für Pfand.
      \item Er/Sie bekommt ein neues Getränk und \textbf{keine} Pfandmarke -- er/sie hat ja von vorher noch eine.
    \end{itemize}
  \item Der Gast ist ein Helfer/Koordinator/Orga:
    \begin{itemize}
      \item Er/Sie gibt die passende Menge Helfermarken ab -- \textbf{keine} Marke extra für Pfand (egal ob ein Becher zurückgegeben wurde oder nicht).
      \item Er/Sie bekommt dafür das Getränk und \textbf{keine} Pfandmarke.
    \end{itemize}
\end{enumerate}
% ---
\subsection{Getränke}
\subsubsection{Bier und Cider}
\begin{itemize}
  \item Bier (Augustiner Hell) in \SI{50}{\litre}-Fässern
  \item Cider (Strongbow) in \SI{30}{\litre}-Fässern
  \item Bier, Radler und Cider werden in \SI{0.5}{\litre}-Bechern ausgeschenkt.
  \item Nach Zapfschluss wird kein neues Fass Bier mehr angebrochen. Stattdessen wird Bier aus den Kästen \textbf{in Becher umgefüllt}. Glasflaschen werden \textbf{nicht} ausgegeben.
\end{itemize}
% ---
\subsubsection{Crown's Energy}
\begin{itemize}
  \item Crown's Energy wird \textbf{nicht} in Dosen ausgegeben, sondern mit etwas Eis im Becher verkauft.
  \item Kostet \textbf{2} Marken, obwohl es Non-Alk ist.
  \item Die zurückgegebenen Dosen kommen in gesonderte Säcke (von Leckerland).
\end{itemize}
% ---
\subsubsection{Non-Alk}
\begin{itemize}
  \item Es gibt Cola, Spezi, Apfelschorle, Zitronenlimo und Wasser (in zwei Sorten: wenig und viel Kohlensäure).
  \item Non-Alk wird \textbf{ohne} Becher in der PET-Flasche verkauft. Wie bei allen Getränken kommt eine Marke für den \textbf{Pfand} dazu und eine Pfandmarke wird ausgegeben.
  \item Es gibt Cola und Wasser in zwei Versionen:
    \begin{itemize}
      \item \SI{0.5}{\litre}-Flasche von Aloisius-Quelle $\rightarrow$ wird als Non-Alk verkauft
      \item \SI{1}{\litre}-Flasche von Coca Cola bzw. Adelholzener $\rightarrow$ wird \textbf{nur} zum Mischen von Rum-Cola bzw.\ Prosecco Aperol verwendet
    \end{itemize}
\end{itemize}
% ---
\subsubsection{Longdrinks}
\begin{enumerate}
  \item ca.\ \SI{100}{\gram} \textbf{Eis} mit der dafür vorgesehenen Schaufel in einen frischen \SI{0.3}{\litre}-Becher
  \item den \textbf{Alkohol} mit dem Dosierer in der richtigen Menge einschenken.
    
    Der Dosierer muss dafür im richtigen Winkel gehalten werden: Hals des Dosierers senkrecht, nicht die Flasche! Dann stoppt der Dosierer nach \SI{5}{\centi\litre}. Um erneut \SI{5}{\centi\litre} einzuschenken, muss die Flasche einmal aufgerichtet werden.
  \item mit \textbf{Non-Alk} auffüllen
  \item einen \textbf{Strohhalm} dazu
\end{enumerate}

\paragraph{Mischverhältnisse}
\begin{center}
  \begin{tabular}{llll}
    Longdrink & Alkohol & Non-Alk & Eis \\ \hline\hline
    Wodka-Orange &  \SI{5}{\centi\litre} Wodka & \SI{15}{\centi\litre} Orangensaft & $\sim$\SI{100}{\gram} \\ \hline
    Wodka-Lemon & \SI{5}{\centi\litre} Wodka & \SI{15}{\centi\litre} Bitter Lemon & $\sim$\SI{100}{\gram} \\ \hline
    Wodka-Energy & \SI{5}{\centi\litre} Wodka & \SI{15}{\centi\litre} Crown's Energy & $\sim$\SI{100}{\gram} \\ \hline
    Crown's Energy & -- & \SI{25}{\centi\litre} (Eine Dose Crown's Energy) & $\sim$\SI{100}{\gram} \\ \hline
    Rum-Cola & \SI{5}{\centi\litre} Rum & \SI{15}{\centi\litre} Coca Cola & $\sim$\SI{100}{\gram} \\ \hline
    Malibu-Kirsch & \SI{10}{\centi\litre} Malibu & 10cl Kirschsaft & $\sim$\SI{100}{\gram} \\ \hline
    Gin-Tonic & \SI{5}{\centi\litre} Gin & \SI{15}{\centi\litre} Tonic Water & $\sim$\SI{100}{\gram} \\ \hline
    Prosecco Aperol & \SI{5}{\centi\litre} Aperol \& \SI{10}{\centi\litre} Proseco & \SI{3}{\centi\litre} Mineralwasser & $\sim$\SI{100}{\gram} \\ \hline
    Hugo & Fertig ca. \SI{20}{\centi\litre} & $\sim$3 Minzblätter
  \end{tabular}
\end{center}

\paragraph{Alkoholsorten}
\begin{itemize}
  \item Wodka: Smirnoff Red Label (\SI{1}{\litre}-Flasche)
  \item Rum: Havanna Club 3 Jahre (\SI{1}{\litre}-Flasche)
  \item Malibu (\SI{1}{\litre}-Flasche)
  \item Gin: Gordon's Dry Gin (\SI{1}{\litre}-Flasche)
  \item Prosecco (\SI{0.7}{\litre}-Flasche)
  \item Aperol (\SI{1}{\litre}-Flasche)
\end{itemize}

% ---
\subsubsection{Cocktails}
\begin{table}[h!]
  \begin{subtable}[t]{0.3\textwidth}
    \centering
    \vspace{0pt}
    \begin{tabular}{|rl|} \hline
      \multicolumn{2}{|l|}{\textbf{Zombie}} \\
      \SI{2}{\centi\litre} & weißer Rum \\
      \SI{2}{\centi\litre} & brauner Rum \\
      \SI{1}{\centi\litre} & Rum, 73\% \\
      \SI{1}{\centi\litre} & Apricot Brandy \\
      \SI{5}{\centi\litre} & Ananassaft \\
      \SI{2}{\centi\litre} & Limettensaft \\
      1 Scheibe & Orange \\
      1 Blatt & Minze \\ \hline
    \end{tabular}
  \end{subtable}
  ~
  \begin{subtable}[t]{0.3\textwidth}
    \centering
    \vspace{0pt}
    \begin{tabular}{|rl|} \hline
      \multicolumn{2}{|l|}{\textbf{Touchdown}} \\
      \SI{4}{\centi\litre} & Wodka \\
      \SI{2}{\centi\litre} & Apricot Brandy \\
      \SI{6}{\centi\litre} & Maracujasaft \\
      \SI{2}{\centi\litre} & Orangensaft \\
      \SI{2}{\centi\litre} & Zitronensaft \\
      \SI{1}{\centi\litre} & Grenadinensirup \\ \hline
    \end{tabular}
  \end{subtable}
  ~
  \begin{subtable}[t]{0.3\textwidth}
    \centering
    \vspace{0pt}
    \begin{tabular}{|rl|} \hline
      \multicolumn{2}{|l|}{\textbf{Sex on the Beach}} \\
      \SI{4}{\centi\litre} & Wodka \\
      \SI{2}{\centi\litre} & Peach Tree Likör \\
      \SI{2}{\centi\litre} & Zitronensaft \\
      \SI{4}{\centi\litre} & Orangensaft \\
      \SI{4}{\centi\litre} & Ananassaft \\
      \SI{1}{\centi\litre} & Grenadinensirup \\ \hline
    \end{tabular}
  \end{subtable}

  \vspace{1cm}
  \begin{subtable}[t]{0.3\textwidth}
    \centering
    \vspace{0pt}
    \begin{tabular}{|rl|} \hline
      \multicolumn{2}{|l|}{\textbf{Jubiläumscocktail}} \\
      \SI{4}{\centi\litre} & brauner Rum \\
      \SI{2}{\centi\litre} & Apricot Brandy \\
      \SI{6}{\centi\litre} & Maracujasaft \\
      \SI{1}{\centi\litre} & Limettensaft \\
      \SI{1}{\centi\litre} & Grenadinensirup \\ \hline
    \end{tabular}
  \end{subtable}
  ~
  \begin{subtable}[t]{0.3\textwidth}
    \centering
    \vspace{0pt}
    \begin{tabular}{|rl|} \hline
      \multicolumn{2}{|l|}{\textbf{Tequila Sunrise}} \\
      \SI{5}{\centi\litre} & Tequila \\
      \SI{10}{\centi\litre} & Orangensaft \\
      \SI{1}{\centi\litre} & Grenadinensirup \\ \hline
    \end{tabular}
  \end{subtable}
  ~
  \begin{subtable}[t]{0.3\textwidth}
    \centering
    \vspace{0pt}
    \begin{tabular}{|rl|} \hline
      \multicolumn{2}{|l|}{\textbf{Mojito}} \\
      \SI{6}{\centi\litre} & weißer Rum \\
      \SI{12}{\centi\litre} & Mineralwasser \\
      \SI{0.5}{EL} & weißer Rohrzucker \\
      \num{0.5} & Limette \\ \hline
    \end{tabular}
  \end{subtable}

  \vspace{1cm}
  \begin{subtable}[t]{0.3\textwidth}
    \centering
    \vspace{0pt}
    \begin{tabular}{|rl|} \hline
      \multicolumn{2}{|l|}{\textbf{Caipirinha}} \\
      \SI{10}{\centi\litre} & Cachaça \\
      \num{1} & Limette \\
      \SI{2}{EL} & brauner Rohrzucker \\
      Rest & Wasser \\ \hline
    \end{tabular}
  \end{subtable}
  ~
  \begin{subtable}[t]{0.3\textwidth}
    \centering
    \vspace{0pt}
    \begin{tabular}{|rl|} \hline
      \multicolumn{2}{|l|}{\textbf{Non-Alk}} \\
      \SI{3}{\centi\litre} & Zitronensaft \\
      \num{1} & Limette \\
      \SI{1.5}{EL} & brauner Rohrzucker \\
      \SI{6}{\centi\litre} & Maracujasaft \\
      Rest & Wasser \\ \hline
    \end{tabular}
  \end{subtable}
  ~
\end{table}
